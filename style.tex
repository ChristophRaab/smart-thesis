%
% Smart Thesis LaTeX template
%
% [To underline the amateurish flavour of this template, let's start with some
% nifty ASCII-Art (http://www.network-science.de/ascii/)...]
%
%      #######
%    /       ###
%   /         ##                                          #
%   ##        #                                          ##
%    ###                                                 ##
%   ## ###      ### /### /###     /###   ###  /###     ########
%    ### ###     ##/ ###/ /##  / / ###  / ###/ #### / ########
%      ### ###    ##  ###/ ###/ /   ###/   ##   ###/     ##
%        ### /##  ##   ##   ## ##    ##    ##            ##
%          #/ /## ##   ##   ## ##    ##    ##            ##
%           #/ ## ##   ##   ## ##    ##    ##            ##
%            # /  ##   ##   ## ##    ##    ##            ##
%  /##        /   ##   ##   ## ##    /#    ##            ##
% /  ########/    ###  ###  ### ####/ ##   ###           ##
%/     #####       ###  ###  ### ###   ##   ###           ##
%|
% \)
%
%
%  /###           /  /
% /  ############/ #/                              #
%/     #########   ##                             ###
%#     /  #        ##                              #
% ##  /  ##        ##
%    /  ###        ##  /##      /##       /###   ###        /###
%   ##   ##        ## / ###    / ###     / #### / ###      / #### /
%   ##   ##        ##/   ###  /   ###   ##  ###/   ##     ##  ###/
%   ##   ##        ##     ## ##    ### ####        ##    ####
%   ##   ##        ##     ## ########    ###       ##      ###
%    ##  ##        ##     ## #######       ###     ##        ###
%     ## #      /  ##     ## ##              ###   ##          ###
%      ###     /   ##     ## ####    /  /###  ##   ##     /###  ##
%       ######/    ##     ##  ######/  / #### /    ### / / #### /
%         ###       ##    ##   #####      ###/      ##/     ###/
%                         /
%                        /
%                       /
%                      /
%
% About:
% ======
%
% This template is a re-implementation of the "classicthesis" template by André
% Miede. However, it uses the "memoir" class as a basis, eliminating most
% of the external packages required by "classicthesis" and thus (hopefully)
% achieving a higher compatibility with other LaTeX packages. Large parts of
% the code have been adapted (stolen) from André Miedes code.
%
% You can find more information about "classicthesis" at CTAN:
%
% https://www.ctan.org/pkg/classicthesis
%
% More information about the "memoir" package can be found here:
%
% https://www.ctan.org/pkg/memoir
%
%
% Authors:
% ========
%
% Jan Philip Göpfert, Andreas Stöckel
%
%
% License:
% ========
%
% This LaTeX template is published under the Creative Commons Zero license. To
% the extent possible under law, the authors have waived all copyright and
% related neighboring rights to Smart Thesis. This work is published from:
% Germany.
%

% Load the texgyrepagella font as main font and set the ``mathpazo'' font as
% math font
\usepackage{fontspec}
\usepackage{mathpazo}
\setmainfont
     [ BoldFont       = texgyrepagella-bold.otf ,
       ItalicFont     = texgyrepagella-italic.otf ,
       BoldItalicFont = texgyrepagella-bolditalic.otf,
       Numbers={OldStyle} ]
     {texgyrepagella-regular.otf}


\renewcommand{\baselinestretch}{1.5}

\usepackage[protrusion=true]{microtype}% Improves type setting. Requires a proper font.
\DeclareRobustCommand{\spacedallcaps}[1]{%
  \addfontfeature{LetterSpace=15,WordSpace=1.25}{\MakeTextUppercase{#1}}%
}%
\DeclareRobustCommand{\spacedlowsmallcaps}[1]{%
  \textsc{\addfontfeature{LetterSpace=15,WordSpace=1.25}{\MakeTextLowercase{#1}}}
}%

\makeatletter
\makechapterstyle{toledo}{%
  \chapterstyle{default}
  \renewcommand*{\chapterheadstart}{}
  \renewcommand*{\printchaptername}{%
    \centerline{\chapnumfont{\@chapapp\ \thechapter}}}
  \renewcommand*{\chapternamenum}{}
  \renewcommand*{\chapnumfont}{\normalfont\scshape\MakeTextLowercase}
  \renewcommand*{\printchapternum}{}
  \renewcommand*{\afterchapternum}{%
    \par\centerline{\parbox{0.5in}{\hrulefill}}\par}
  \renewcommand*{\printchapternonum}{%
    \vphantom{\chapnumfont \@chapapp 1}\par 
    \parbox{0.5in}{}\par}
  \renewcommand*{\chaptitlefont}{\normalfont\large}
  \renewcommand*{\printchaptertitle}[1]{%
    \centering \chaptitlefont\spacedallcaps{##1}}}

\nouppercaseheads
\makepagestyle{berlin}
\makeevenfoot{berlin}{\thepage}{}{}
\makeoddfoot{berlin}{}{}{\thepage}

\copypagestyle{chapter}{berlin}

%\if@twoside
    \makepsmarks{berlin}{%
      \def\chaptermark##1{%
        \markboth{\memUChead{%
          \ifnum \c@secnumdepth >\m@ne
            \if@mainmatter
              \@chapapp\ \thechapter. \ %
            \fi
          \fi
          ##1}}{}}%
      \def\tocmark{\markboth{\memUChead{\contentsname}}{\memUChead{\contentsname}}}%
      \def\lofmark{\markboth{\memUChead{\listfigurename}}{\memUChead{\listfigurename}}}%
      \def\lotmark{\markboth{\memUChead{\listtablename}}{\memUChead{\listtablename}}}%
      \def\bibmark{\markboth{\memUChead{\bibname}}{\memUChead{\bibname}}}%
      \def\indexmark{\markboth{\memUChead{\indexname}}{\memUChead{\indexname}}}%
      \def\sectionmark##1{%
        \markright{\memUChead{%
          \ifnum \c@secnumdepth > \z@
            \thesection. \ %
          \fi
          ##1}}}}
    \makepsmarks{berlin}{%
      \createmark{chapter}{left}{nonumber}{}{}
      \createmark{section}{right}{shownumber}{}{. \ }
      \createplainmark{toc}{both}{\contentsname}
      \createplainmark{lof}{both}{\listfigurename}
      \createplainmark{lot}{both}{\listtablename}
      \createplainmark{bib}{both}{\bibname}
      \createplainmark{index}{both}{\indexname}
      \createplainmark{glossary}{both}{\glossaryname}
    }
    \makeevenhead{berlin}{\spacedlowsmallcaps{\leftmark}}{}{}
    \makeoddhead{berlin}{}{}{\spacedlowsmallcaps{\rightmark}}
%\else
%    \makepsmarks{berlin}{%
%      \def\chaptermark##1{%
%        \markright{\memUChead{%
%          \ifnum \c@secnumdepth >\m@ne
%            \if@mainmatter
%              \@chapapp\ \thechapter. \ %
%            \fi
%          \fi
%          ##1}}}%
%      \def\tocmark{\markright{\memUChead{\contentsname}}}%
%      \def\lofmark{\markright{\memUChead{\listfigurename}}}%
%      \def\lotmark{\markright{\memUChead{\listtablename}}}%
%      \def\bibmark{\markright{\memUChead{\bibname}}}%
%      \def\indexmark{\markright{\memUChead{\indexname}}}}
%    \makepsmarks{berlin}{%
%      \createmark{chapter}{right}{shownumber}{\@chapapp\ }{. \ }
%      \createplainmark{toc}{right}{\contentsname}
%      \createplainmark{lof}{right}{\listfigurename}
%      \createplainmark{lot}{right}{\listtablename}
%      \createplainmark{bib}{right}{\bibname}
%      \createplainmark{index}{right}{\indexname}
%      \createplainmark{glossary}{right}{\glossaryname}
%    }
%    \makeoddhead{berlin}{\slshape\rightmark}{}{}
%\fi

\makeatother

\chapterstyle{toledo}
\pagestyle{berlin}

\strictpagecheck
\newcommand{\marginnote}[1]{\marginpar{
  \renewcommand{\baselinestretch}{1.25}
  \small
  \itshape
  \checkoddpage
  \ifoddpage
    \raggedright
  \else
    \raggedleft
  \fi
  {#1}
}}

%
% FOOTNOTES
%

\footmarkstyle{\textsc{#1}}
\setlength{\footmarkwidth}{-0.5em}
\setlength{\footmarksep}{0em}
\setlength{\footparindent}{0em}

%
% EPIGRAPHS
%
\makeatletter

% Make the epigraph span two third of the text width
\setlength{\epigraphwidth}{0.66\textwidth}

% Disable the rule below the epigraph
\setlength{\epigraphrule}{0pt}

% Use the normal font size
\epigraphfontsize{\normalsize}

% Style the epigraph itself -- use italic text and
% flush it to the right
\newenvironment{epigraphstyle}{\slshape\raggedleft}{}
\epigraphtextposition{epigraphstyle}

% Style the epigraph source -- prepend it with a em-dash
% and make sure no indentation is added after the epigraph
\newenvironment{epigraphsourcestyle}%
{%
	--- \raggedleft\color{webgreen}%
}{%
	\@afterindentfalse\@afterheading%
}
\epigraphsourceposition{epigraphsourcestyle}

\makeatother

%
% PAGE LAYOUT
%
\settypeblocksize{23.45cm}{11.85cm}{*}
\setulmargins{*}{*}{*}
\setmarginnotes{0.8cm}{3.3cm}{1cm}
\checkandfixthelayout

